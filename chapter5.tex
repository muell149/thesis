%
% Chapter 5
%

\chapter{EVENT SELECTION}
The event selection process defines the signal regions of this analysis. The signal regions are categories that contain events passing a series of cuts desinged to 
select as much signal (\tth) as possible, and simultaneously reject as much background as possible, therefore every cut defining each signal region chosen with this
in mind. In the multilepton analysis, there are three signal regions predominantly defined by the lepton multiplicity in the event: the two-lepton same-sign ($2lss$) region,
the three-lepton ($3l$) region, and the four-lepton ($4l$) region. This dissertation focuses $2lss$ category. Defining the signal regions by lepton multiplicity is motivated
by the fact that the number of leptons passing the object selection yields the most information about the underlying event.

There are several cuts which don't depend on lepton multiplicity that are applied to all signal regions to ensure the selection of events consistent with \tth. Events with
a pair of loose leptons with an invariant mass of less than 12 GeV are vetoed, as they are consistent with a J/$\psi$ or $\Upsilon$ decay and are not modeled in the MC
simulation, but are present in data. At least two jets are required in all signal regions, and of these there must be at least one jet passing the medium CSVv2 working point,
or two passing the loose working point. This b-jet requirement is consistent with a top-quark pair decaying to jets, which is present in all \tth processes. 

\section{Two-lepton same-sign category}
The $2lss$ category is primarily defined by the requirement that there be exactly two tight leptons that have the same sign electric charge. The same-sign requirement is needed to veto 
one of the largest backgrounds dileptonic \ttbar + jets, which has oppositely charged leptons and a cross section more than three orders of magnitude greater than that of \tth. We require
the leading lepton have \pt $>$ 25 GeV to stay well above the trigger threshold for reasons that will be explained in the trigger section. At least 4 jets are required, to be consistent with
a \tth process with two same-sign leptons in the final state. To reject backgrounds with Z bosons where one of the lepton charges is mis-measured, we reject any di-electron pair whose invariant
mass is within 10 GeV of the Z mass (90.2 GeV) and also require the \met LD be greater than 0.2 (again, for di-electron events only). To summarize, the $2lss$ category is defined by events
satisfying the collowing requirements:

\begin{itemize}
 \item Exactly two tight leptons with the same-sign electric charge and \pt $>$ 25,15 GeV
 \item m($ll$) $<$ 12 GeV for any pair of loose leptons in the event
 \item $\geq$4 jets, among which there must be $\geq$1 CSVv2 M or $\geq$2 CSVv2 L
 \item $|m(ee)-m_{Z}| >$ 10 GeV (for ee events only)
 \item \met LD $>$ 0.2 (for ee events only) 
\end{itemize}

%\section{Three and four-lepton categories}
%% \begin{figure}[hbtp]
%%  \begin{center}
%%    \includegraphics[width=0.8\textwidth]{ch4_figs/cms_particleflow.pdf}
%%    \caption{An overview of how CMS detects different types of particles. The slice of CMS in in the x-y plane.~\cite{NEED CITATION}.}
%%    \label{fig:cms_pflow}
%%  \end{center}
%% \end{figure}
