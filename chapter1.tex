%
% Chapter 1
%

\chapter{INTRODUCTION}
The recent observation of the Higgs boson confirmed the mechanism by which matter acquires mass. While this discovery made
significant progress towards completing the Standard Model's overall successful description of nature, many important questions
regarding the origins of mass remain unanswered. Do the observed properties of the Higgs boson match SM expectations? Why do
particles have the specific masses we observe? Is the top quark's large mass coming solely from its interaction with the Higgs?
Answering these questions will provide crucial insight into the underlying principles that govern our universe. The research and
analysis presented here attempts to address these profound questions. 

This analysis aims to discover processes from proton-proton collisions at the LHC where a Higgs boson is produced in association
with a top-antitop quark pair (denoted as \tth) and decays to final states with two or more charged leptons in the CMS detector
at a center-of-mass collision energy of 13 TeV.
Referred to as \tth multilepton processes, these provide an efficient probe with which to test the Standard Model. 

NOT SURE IF DISCUSSION OF PREVIOUS \tth RESULTS IS BETTER HERE OR AS A PROLOGUE TO RESULTS CHAPTER...

% % uncomment the following lines,
% if using chapter-wise bibliography
%
% \bibliographystyle{ndnatbib}
% \bibliography{example}
