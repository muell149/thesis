%
% Modified by Megan Patnott
% Last Change: Jan 18, 2013
%
%%%%%%%%%%%%%%%%%%%%%%%%%%%%%%%%%%%%%%%%%%%%%%%%%%%%%%%%%%%%%%%%%%%%%%%%
%
% Modified by Sameer Vijay
% Last Change: Tue Jul 26 2005 13:00 CEST
%
%%%%%%%%%%%%%%%%%%%%%%%%%%%%%%%%%%%%%%%%%%%%%%%%%%%%%%%%%%%%%%%%%%%%%%%%
%
% Sample Notre Dame Thesis/Dissertation
% Using Donald Peterson's ndthesis classfile
%
% Written by Jeff Squyres and Don Peterson
%
% Provided by the Information Technology Committee of
%   the Graduate Student Union
%   http://www.gsu.nd.edu/
%
% Nothing in this document is serious except the format.  :-)
%
% If you have any suggestions, comments, questions, please send e-mail
% to: ndthesis@gsu.nd.edu
%
%%%%%%%%%%%%%%%%%%%%%%%%%%%%%%%%%%%%%%%%%%%%%%%%%%%%%%%%%%%%%%%%%%%%%%%%


%
% Chapter 2
%

\chapter{THEORETICAL BACKGROUND}
\section{Introduction to the Standard Model}
The Standard Model (SM) of particle physics is one of the most elegant descriptions of nature available. It explains three of four fundamental forces via gauge symmetries, while characterizing unknown matter into
separate generations of particles called quarks and leptons. Since its inception in the early 1960s, the SM has predicted the existence of every fundamental particle that has been discovered to-date.
The SM distills the real-world observables matter and energy into discrete elementary particles and their kinematics. The SM is the theory on which all of the following research is based, and also the theory and hypothesis that is being tested.

\subsection{Particles and Fields}
The particles in the SM are first characterized by their intrinisic angular momentum, more commonly referred to as spin. Particles with half-integer spin, quantized in units of Planck's constant $\hbar$, are fermions, while particles with integer spin are bosons.
This distinction is important because the spin values govern behavior and interactions of the statistics of sets of collections of particles. 

The fermions in the SM are the most fundamental examples of matter in nature. Fermions behave and interact according to Fermi-Dirac statistics, and obey the Pauli exclusion principle. Fermions are further categorized, based on their primary interaction mechanism,
into quarks and leptons. There are six different flavors of quarks in the SM, the up and down, the charm and strange, and the top and bottom quarks are organized into three generations of doublets below.
\begin{equation}
\binom{u}{d} \;\;\; \binom{c}{s} \;\;\; \binom{t}{b}
\end{equation}
\noindent An increasing mass (from left to right) distinguishes each generation, while the upper and lower elements of each doublet are distinguished by an electric charge of +2/3 and -1/3 respectively in each generation. Quarks interact via the strong, weak, and
electromagnetic interactions. Quarks also carry a color charge, which can assume one of three values (red, blue, green) as a result of the strong interaction described by Quantum Chromodynamics (QCD). The leptons in the SM can also be arrange into three increasingly
massive generations of doublets.
\begin{equation}
\binom{e^{-}}{\nu_{e}} \;\;\; \binom{\mu^{-}}{\nu_{\mu}} \;\;\; \binom{\tau^{-}}{\nu_{\tau}}
\end{equation}
\noindent The upper elements in each lepton doublet are the familiar electron, and the less familiar but much heavier, muon and tau. Due to their increased mass, the muon and tau have very short lifetimes which causing them decay rapidly
to lighter, more stable particles. The electron, muon, and tau all have the same electric charge of -1. The lower elements in each doublet are the lightweight and electrically neutral counterparts called neutrinos, which also come in three
flavors; the electron-neutrino, the muon-neutrino, and the tau-neutrino. While the electron, muon, and tau interact via both the electromagnetic and weak force, neutrinos interact only through the weak force. Neutrinos are characterized by
how weakly they interact. They interact so weakly that they are able to pass through all of planet earth with a single interaction! This unfortunate property makes it impossible to directly detect their presence at CMS. 
   


\subsection{Gauge Symmetries}
\subsection{Electroweak Symmetry Breaking}
\section{The Higgs Boson}

\section{ttH}


% % uncomment the following lines,
% if using chapter-wise bibliography
%
% \bibliographystyle{ndnatbib}
% \bibliography{example}
