%
% Modified by Megan Patnott
% Last Change: Jan 18, 2013
%
%%%%%%%%%%%%%%%%%%%%%%%%%%%%%%%%%%%%%%%%%%%%%%%%%%%%%%%%%%%%%%%%%%%%%%%%
%
% Modified by Sameer Vijay
% Last Change: Tue Jul 26 2005 13:00 CEST
%
%%%%%%%%%%%%%%%%%%%%%%%%%%%%%%%%%%%%%%%%%%%%%%%%%%%%%%%%%%%%%%%%%%%%%%%%
%
% Sample Notre Dame Thesis/Dissertation
% Using Donald Peterson's ndthesis classfile
%
% Written by Jeff Squyres and Don Peterson
%
% Provided by the Information Technology Committee of
%   the Graduate Student Union
%   http://www.gsu.nd.edu/
%
% Nothing in this document is serious except the format.  :-)
%
% If you have any suggestions, comments, questions, please send e-mail
% to: ndthesis@gsu.nd.edu
%
%%%%%%%%%%%%%%%%%%%%%%%%%%%%%%%%%%%%%%%%%%%%%%%%%%%%%%%%%%%%%%%%%%%%%%%%


%
% Chapter 2
%

\chapter{THEORETICAL BACKGROUND}
\section{Introduction to the Standard Model}
The Standard Model (SM) of particle physics is one of the most elegant descriptions of nature available. It explains three of four fundamental forces via gauge symmetries, while characterizing unknown matter into
separate generations of particles called quarks and leptons. Since its inception in the early 1960s, the SM has predicted the existence of every fundamental particle that has been discovered to-date.
The SM distills the real-world observables matter and energy into discrete elementary particles and their kinematics. The SM is the theory on which all of the following research is based, and also the theory and hypothesis that is being tested.

\subsection{Particles and Fields}
The particles in the SM are first characterized by their intrinisic angular momentum, more commonly referred to as spin. Particles with half-integer spin, quantized in units of Planck's constant $\hbar$, are fermions, while particles with integer spin are bosons.
This distinction is important because the spin values govern behavior and interactions of the statistics of sets of collections of particles. 

The fermions in the SM are the most fundamental examples of matter in nature. Fermions behave and interact according to Fermi-Dirac statistics, and obey the Pauli exclusion principle. Fermions are further categorized, based on their primary interaction mechanism,
into quarks and leptons. There are six different flavors of quarks in the SM, the up and down, the charm and strange, and the top and bottom quarks are organized into three generations of doublets below.
\begin{equation}
\binom{u}{d} \;\;\; \binom{c}{s} \;\;\; \binom{t}{b}
\end{equation}
\noindent An increasing mass (from left to right) distinguishes each generation, while the upper and lower elements of each doublet are distinguished by an electric charge of +2/3 and -1/3 respectively in each generation. Quarks interact via the strong, weak, and
electromagnetic interactions. Quarks also carry a color charge, which can assume one of three values (red, blue, green) as a result of the strong interaction described by Quantum Chromodynamics (QCD). The leptons in the SM can also be arrange into three increasingly
massive generations of doublets.
\begin{equation}
\binom{e^{-}}{\nu_{e}} \;\;\; \binom{\mu^{-}}{\nu_{\mu}} \;\;\; \binom{\tau^{-}}{\nu_{\tau}}
\end{equation}
\noindent The upper elements in each lepton doublet are the familiar electron, and the less familiar but much heavier, muon and tau. Due to their increased mass, the muon and tau have very short lifetimes which causes them decay rapidly
to lighter, more stable particles. The electron, muon, and tau all have the same electric charge of -1. The lower elements in each doublet are the lightweight and electrically neutral counterparts called neutrinos, which also come in three
flavors; the electron-neutrino, the muon-neutrino, and the tau-neutrino. While the electron, muon, and tau interact via both the electromagnetic and weak force, neutrinos interact only through the weak force. Neutrinos are characterized by
how weakly they interact. They interact so weakly that they are able to pass through all of planet earth without a single interaction. This property makes it impossible to directly detect their presence at CMS. 

\begin{figure}[hbtp]
 \begin{center}
   \includegraphics[width=0.7\textwidth]{sm_particles_periodic_table.pdf}
   \caption[text in square brackets]{A sumamry of all elementary particles and their interactions in the Standard Model.}
   \label{fig:sm_periodic_table}
 \end{center}
\end{figure}


The bosons in the SM are also fundamental, but are not examples of matter. Bosons are characterized by their integer-quantized angular momentum and behave according to Bose-Einstein statistics. There are five elementary SM bosons, the four force-carrying gauge
bosons, and one scalar (spin-0) boson that was recently discovered in 2012, known as the Higgs boson. The four forces (strong, weak, electromagnetic, gravity) through which particles interact are all carried by the corresponding gauge bosons, with the exception
of gravity. The SM currently does not explain gravity because there is no SM particle that carries its force. The gauge boson believed to be responsible for this, the graviton, has yet to be discovered and likely exists in a mass range far beyond the reach of
the LHC. The strongest of the four forces, the appropriately-named strong force is carried by the gluon. Gluons are spin-0, electrically neutral, massless, and carry a color charge. Gluons mediate the strong force through which quarks interact. Due to the nature of
color charge and confinement, gluons keep the quarks inside protons and neutrons glued together, confined inside the proton. Additionally, strong force also binds protons and neutrons together to form nuclei of atoms. Any particle carrying color charge is capable
of strong interactions. 
The photon is the gauge boson that mediates the next strongest interaction, the electromagnetic force. The photon is massless, spin-1, electrically neutral, and travels at the speed of light. Aside from gravity, the electromagnetic force is the most familiar,
responsible for keeping electron orbitals bound to nuclei, forming atoms, and it is also responsible for the attractive and repulsive forces that bond atoms together into molecules. Any particle carrying electric charge is capable of interacting electromagnetically.
The weakest force explained by the SM, the appropriately named weak force is mediated by the massive W and Z gauge bosons. There are two types of weak interactions, charged and neutral.
There are two W bosons, W$^+$,W$^-$ which identical except differ by electric charges of +1 and -1 respectively. The spin-1 W boson has a mass of 80.4 GeV, and mediates the weak charged interaction.
The electrically neutral, spin-1 Z boson weighs 91.2 GeV and mediates the weak neutral interaction. The nuclear decay of unstable atoms, which is harnessed for nuclear power, is possible thanks to the weak interaction.
The final elementary SM boson is the Higgs. The Higgs is a massive spin-0, electrically neutral boson that interacts with both fermions and other bosons and weighs approximately 125 GeV. While the Higgs doesn't mediate a force, it does represent a field, and the
mechanism by which elementary particles obtain masses. The Higgs boson and the origins of mass will be explained the coming sections. 

\subsection{Symmetries of the Standard Model}
\subsection{Electroweak Symmetry Breaking}
\section{The Higgs Boson}

\section{ttH}


% % uncomment the following lines,
% if using chapter-wise bibliography
%
% \bibliographystyle{ndnatbib}
% \bibliography{example}
