%
% Chapter 2
%

\chapter{THEORY}
\label{chap:theory}
\section{The Standard Model}
\label{sec:sm}
The Standard Model (SM) of particles physics provides the context in which this analysis is performed and the results interpreted. 
It explains three of four fundamental forces via gauge symmetries, while characterizing unknown matter into
separate generations of particles called quarks and leptons. Since its inception in the early 1960s, the SM has predicted the existence of nearly every fundamental particle that has been discovered to-date.
The SM distills the real-world observables matter and energy into discrete elementary particles and their kinematics. The SM is the theory on which all of the following research is based, and also the hypothesis being tested in this analysis.

\subsection{Particle Content}
The particles in the SM are first characterized by their intrinisic angular momentum, more commonly referred to as spin. Particles with half-integer spin, quantized in units proportional to Planck's constant $\hbar$, are fermions, while particles with integer spin are bosons.
This distinction is important because the spin values govern behavior and interactions of collections of particles. 

The fermions in the SM are the most fundamental examples of matter in nature. Fermions behave and interact according to Fermi-Dirac statistics, and obey the Pauli exclusion principle. Fermions are further categorized, based on their primary interaction mechanism,
into quarks and leptons. There are six different flavors of quarks in the SM, the up and down, the charm and strange, and the top and bottom quarks are organized into three generations of doublets below.
\begin{equation}
\binom{u}{d} \;\;\; \binom{c}{s} \;\;\; \binom{t}{b}
\end{equation}
\noindent An increasing mass (from left to right) distinguishes each generation, while the upper and lower elements of each doublet are distinguished by an electric charge of +2/3 and -1/3 respectively in each generation. Quarks interact via the strong and
electroweak forces. Quarks also carry a color charge, which can assume one of three values (red, blue, green) as a result of the strong interaction described by Quantum Chromodynamics (QCD). The leptons in the SM can also be arranged into three increasingly
massive generations of doublets.
\begin{equation}
\binom{e^{-}}{\nu_{e}} \;\;\; \binom{\mu^{-}}{\nu_{\mu}} \;\;\; \binom{\tau^{-}}{\nu_{\tau}}
\end{equation}
\noindent The upper elements in each lepton doublet are the familiar electron, and the less familiar but much heavier, muon and tau. Due to their increased mass, the muon and tau have short lifetimes which causes them decay rapidly to lighter, more stable particles. In the context of CMS however, the muon is stable. The taus decay inside CMS and the
subsequent reconstruction will be discussed in the following chapters. 
The electron, muon, and tau all have the same electric charge of -1.

The lower elements in each doublet are the lightweight and electrically neutral counterparts called neutrinos, which also come in three
flavors; the electron-neutrino, the muon-neutrino, and the tau-neutrino. Neutrinos interact primarily through the weak force. In an experimental context such as CMS, neutrinos are characterized by
how weakly they interact. They effectively don't interact at all for what concerns CMS. They interact so weakly that they are able to pass through all of planet earth without a single interaction.
This property makes it impossible to directly detect their presence at CMS. For every electrically charged example of matter described above, there exists a nearly identical anti-matter version. Antimatter is identical to matter, except that the
signs on all charges and spins are opposite that of ordinary matter. Matter and anti-matter interact via the same forces/gauge bosons. 

\begin{figure}[hbtp]
 \begin{center}
   \includegraphics[width=0.8\textwidth]{ch2_figs/sm_particles_periodic_table.pdf}
   \caption[Particle content of the SM]{A summary of all elementary particles and their interactions in the Standard Model~\cite{sm_table}.}
   \label{fig:sm_periodic_table}
 \end{center}
\end{figure}

The bosons in the SM are also fundamental, but are not examples of matter. Bosons are characterized by their integer-quantized (in units proportional to $\hbar$) angular momentum and behave according to Bose-Einstein statistics. There are five elementary SM bosons, the four force-carrying gauge
bosons, and one scalar (spin-0) boson that was recently discovered in 2012, known as the Higgs boson. Three of the four forces (strong, weak, electromagnetic) through which particles interact are all
carried by corresponding gauge bosons. The SM currently does not explain or incorporate gravity and as such there is no SM particle that carries its force. The hypothetical gauge boson believed to be responsible for gravity is the graviton,
and has yet to be discovered because its coupling to SM particles is too weak to be probed by the current center-of-mass energies that today's colliders are capable of producing.
The strongest of the four forces, the appropriately-named strong force is carried by the gluon. Gluons are spin-1, electrically neutral, massless, and carry a color charge. Gluons mediate the strong force through which quarks interact. Due to the nature of
color charge and confinement, gluons keep the quarks \emph{glued} together, confined inside hadrons. Additionally, the strong force also binds protons and neutrons together to form nuclei of atoms. Any particle carrying color-charge
is capable of strong interactions. 
The photon is the gauge boson that mediates the next strongest interaction, the electromagnetic force. The photon is massless, spin-1, electrically neutral, and travels at the speed of light. Aside from gravity, the electromagnetic force is the most familiar,
responsible for keeping electron orbitals bound to nuclei, forming atoms, and it is also responsible for the attractive and repulsive forces that bond atoms together into molecules. Any particle carrying electric charge is capable of interacting electromagnetically.
The weakest force explained by the SM, the appropriately named weak force is mediated by the massive W and Z gauge bosons. There are two types of weak interactions, charged and neutral.
There are two W bosons, W$^+$,W$^-$ which are identical except for their electric charges of +1 and -1 respectively. The spin-1 W boson has a mass of 80.4 GeV~\cite{pdg}, and mediates the weak charged interaction.
The electrically neutral, spin-1 Z boson has a mass of 91.2 GeV and mediates the weak neutral interaction~\cite{pdg}. The decay of unstable atoms, which is harnessed for nuclear power, is possible thanks to the weak interaction.
The final elementary SM boson is the Higgs. The Higgs is a massive spin-0, electrically neutral boson that interacts with both fermions and other bosons and has a mass of approximately 125 GeV~\cite{pdg}. While the Higgs doesn't mediate a force,
it does represent the Higgs field and the
mechanism by which elementary particles (matter) obtain masses. The Higgs boson and the origins of mass will be explained the coming sections. 

\begin{table}[hbtp]
\centering
\caption[Relative strengths of the fundamental forces]{Relative dimensionless strengths of the fundamental forces}
\begin{tabular}{lcc}
\hline
Force & Relative Strength \\
\hline
Gravity & $1$ \\
Weak &  $10^{25}$ \\
Electromagnetic & $10^{36}$ \\
Strong & $10^{38}$ \\
\hline
\end{tabular}
\label{tab:force_table}
\end{table}


\subsection{Fields and Symmetries}
The notion of symmetry is of central importance to the SM.
The SM describes three of the fundamental forces with gauge field theories.
A gauge theory is a field theory whose Lagrangian is invariant under a specific type
of transformation, called a gauge transformation. These transformations form a symmetry (group), because the Lagrangian is invariant under the transformation.
The word symmetry is used to describe these transformations because
the underlying dynamics of the theory are left unchanged or \emph{symmetric} with respect to the transformation.
Thanks to Noether's Theorem, these symmetries are critical to the SM because they describe and invoke the convservation laws that SM particles obey.
Each force within the SM corresponds to its own gauge field theory
and symmetry that describe it. In each gauge theory, the corresponding field or force-carrying gauge boson is represented by the generators of the symmetry group. 

Originally, the gauge theory that described the electromagnetic force, Quantum Electrodynamics (QED), was invariant under a $U(1)$ symmetry that described the photon as the
force-carrying gauge boson and the symmetry required electric charge to be conserved. QED was first developed by Sinitoro Tomononga, Julian Schwinger, and Richard Feynman.
This laid the foundation on which the rest of the SM was built and won them the Nobel Prize in 1965~\cite{NP65}. 

Not long after the development of QED, C.N. Yang and Robert Mills formalized the gauge field theory techniques that were further refined by Sheldon Glashow, Abdus Salam,
and Steven Weinberg in the 1960s to combine QED
with the description of the weak force, producing a unified electroweak theory that described electromagnetism and the weak force~\cite{NP79}.
%%continueing paragraph
With this unification came new quantum numbers and conservation laws associated with the electroweak force. Among them are the conserved quantities of isospin and hypercharge\footnote{Also commonly
referred to as weak isospin and weak hypercharge}. These conserved quantites are actually used to redefine the already-conserved and familiar quantity of electric charge

\begin{equation}
\label{eqn:hypercharge}
Q = T_{3} + \frac{Y_{W}}{2}
\end{equation}
 
\noindent where $Q$ is the familiar electric charge, $T_{3}$ is the third component of weak isospin and $Y_{W}$ is the hypercharge. With these newly introduced quantum numbers,
the notion of chirality or handedness is now very relevant
in describing the electroweak force. Chirality is defined as the orientation of a particle's spin vector with respect to its linear momentum vector.
Particles are said to be lefthanded when their spin vector is antialigned with
their linear momentum vector, and have $T_{3}=\pm 1/2$, and righthanded when the two vectors are aligned, and have $T_{3}=0$.  
This unified electroweak theory is a
$U_{Y}(1) \otimes SU_{L}(2)$ gauge theory, where Y represents the weak hypercharge, and L means that only left-handed fermions participate in the weak interaction. 
The corresponding gauge bosons are the massless photon, and the massive weak force-carrying W and Z bosons.
The initial shortcoming of electroweak theory was that it lacked an explanation for the broken symmetry:
it didn't describe why the W and Z gauge bosons had mass in contrast to the massless gauge boson, the photon, from QED. 
The explanation for and mechanism behind this electroweak symmetry breaking is described in the following subsection.

The remaining gauge theory necessary for the SM is QCD. The theoretical underpinnings of QCD were developed in 1965 by Moo-Young Han, Yoichiro Nambu, and Oscar Greenberg, after
significant work from Murray Gell-Mann and others describing the interactions of quarks~\cite{NP69}.  
Unlike the QED, QCD is a non-abelian gauge field theory. The consequence of this is that the force-carrying gauge boson
of QCD, the gluon, can interact with itself (and other gluons). This non-abelian gauge theory is denoted as $SU_{c}(3)$, with the conserved property of QCD being color charge.
Bare color-charged particles
cannot exist alone, rather they are confined to color-neutral states. Everyday examples of this include baryons (which include protons and neutrons) where each constituent quark
carries a unique color charge. At very short distances and very high energies however, it is possible to observe an approximately unconfined quark, however the strong confining
forces increases with increasing distance between color charges until it becomes energetically favorable to generate a new color-neutral quark-antiquark pair from the vacuum.
This process, known as fragmentation\footnote{Also referred to as hadronization.}, has very important consequences in the context of experimental physics at CMS, as the hard-scatter LHC
collisions often produce bare quarks which immediately hadronize into jets, or collimated sprays of energetic particles.
The experimental techniques used to detect and reconstruct these jets are described in the Section~\ref{sec:cms}. 

Combining electroweak theory with QCD, we have the SM, an $U_{Y}(1) \otimes SU_{L}(2) \otimes SU_{c}(3)$ gauge field theory that describes all interactions and particles in
Figure~\ref{fig:sm_periodic_table}. The remaining unexplained portions of electroweak theory, namely the Higgs mechanism and its implications on particle masses are explained next.

\subsection{Electroweak Symmetry Breaking}

In electroweak $U_{Y}(1) \otimes SU_{L}(2)$ gauge theory, gauge invariance requires all of the force-carrying gauge bosons to be massless.
For everything to make sense, the weak force-carrying W and Z gauge bosons would need to be massless, just like the photon. 
However observations indicate that the W and Z are massive, along with the fermions.
So why are the W and Z massive and the photon massless? Where do fermions get their mass?
Enter electroweak spontaneous symmetry breaking (EWSB). Thanks to the identification of EWSB, in 1964 
three groups almost simultaneously explained the origins of particle masses using EWSB and gauge invariance in what became known as the 1964 PRL symmetry breaking papers
\cite{1964_prl_englert}\cite{1964_prl_higgs}\cite{1964_prl_guralnik}.
The explanation by which particles aquire masses is the Higgs Mechanism. For the particles in electroweak theory to have mass, they must be coupled\footnote{The phrases
``coupling to'' and ``interacting with'' have the same meaning and are used interchangeably.} to a field.
Adding the Higgs field to the electroweak Lagrangian and imposing gauge invariance gives masses to the W and Z gauge bosons as well as the fermions, but it does not interact
or give mass to the photon. Giving mass to the W and Z but not the photon spontaneously breaks the $U_{Y}(1) \otimes SU_{L}(2)$ symmetry.
Adding the Higgs field to the electroweak Lagrangian must be done in such a way as to minimize the new potential. The addition of the Higgs field means that new potential
is minimized at a non-zero value. This non-zero value is referred to as the vacuum expectation value (V.E.V.) and is the ground state of the SM.
The fields in the SM are considered fluctuations around the V.E.V. The addition of the Higgs field and minimization of the potential to the resulting ground state
is said to spontaneously break the symmetry because no external impetus for it exists. That is, there are many ways to break the symmetry and one is chosen by nature at random.
A famous and more intuitive example of spontaneous symmetry breaking can be illustrated
by imagining a plastic ruler held vertically between your hands with the skinny edge (not the face) facing you and then compressing your hands so that the center of the ruler bows to the right or the left and
spontaneously breaks the right left symmetry of the ruler-hand system~\cite{robinson}.
This Higgs field, similar to the gauge fields, is represented by a particle called the Higgs boson.
For the field to give all electroweak particles mass, it interacts with bosons and fermions.
The interaction of the Higgs boson (or any boson) with fermions is known as the Yukawa interaction or coupling. Measuring the strength of the Yukawa coupling
of the Higgs boson with the top quark is the primary focus of this dissertation.

\subsection{The Higgs Boson}

The particle manifestation of the Higgs field is the Higgs boson. After the Higgs mechanism was first introduced in 1964 to explain the origins of particle mass,
the race was on to experimentally confirm the massive gauge bosons predicted by the unified electroweak gauge theory including the Higgs mechanism ~\cite{1964_prl_higgs}.
The W and Z were discovered and their masses confirmed in 1983 by the UA1 and UA2 experiments at the Super Proton Synchrotron (SPS) at CERN~\cite{UA1}\cite{UA2}\cite{Z}.

The missing piece of the puzzle was the Higgs boson. The Large Electron Positron Collider (LEP) at CERN was the next place to look for the Higgs.
Beginning in 1989, experiments at LEP searched for the Higgs at center-of-mass energies ranging from 45 GeV early on, to over 200 GeV in 2000 \cite{LEPHIGGS}. Meanwhile, Higgs
searches were also being conducted at the Tevatron collider at Fermilab. The Tevatron reached higher collision energies than LEP, with its second run lasting from 2001
to 2011, colliding protons and anti-protons at a center-of-mass energy 10x greater than that of LEP, reaching nearly 2 TeV \cite{TEVHIGGS_2010}. The next collider, the Large Hadron Collider
(LHC) at CERN came online in 2008. The search for the Higgs came to an end when the ATLAS and CMS experiments announced the discovery of the Higgs in 2012 at the LHC~\cite{cms_higgs}\cite{atlas_higgs}. 

Almost 48 years after being theorized, the Higgs discovery proved the existence
of the Higgs field, and validated the Higgs mechanism as well as the SM. The 2013 Nobel Prize was awarded to
Peter Higgs and Francois Englert\footnote{Robert Brout contributed equally to this work, but died in 2011.} for their explanation of the origins of mass\footnote{This is
technically known and published as the BEH mechanism after the initial authors~\cite{1964_prl_higgs}~\cite{1964_prl_englert}.}~\cite{NP13}.

While this Higgs discovery is important, it is more relevant now to address the production mechanisms that made it possible. The LHC collides beams of protons together;
however, it is the quarks and gluons inside the protons that are actually colliding. The most common processes that produce Higgs bosons at the LHC are described in the
Feynman diagrams in Figure~\ref{fig:higgs_production}. Of these processes, gluon fusion is the most common and also the mode targeted and used by the analyses that
discovered the Higgs. Associated (Higgs) production is the mechanism that produces the Higgs processes studied in this dissertation, and occurs much less frequently compared to
the other production modes. These cross sections of each process are shown in Figure~\ref{fig:higgs_prod_plot}.


\begin{figure}[htbp] 
  {\centering
    \subfigure[gluon fusion]{\includegraphics[width=0.4\textwidth]{ch2_figs/ggH.pdf}}
    \subfigure[vector boson fusion]{\includegraphics[width=0.4\textwidth]{ch2_figs/VBFH.pdf}}
    \subfigure[Higgs-strahlung]{\includegraphics[width=0.4\textwidth]{ch2_figs/VH.pdf}}
    \subfigure[associated production]{\includegraphics[width=0.4\textwidth]{ch2_figs/ttH.pdf}}
    \caption[Higgs boson production modes at the LHC]{Higgs boson production modes at the LHC:
      gluon fusion $gg\to{}H$ (a), vector boson fusion $qq\to{}qqH$ (b),
      Higgs-strahlung $q\bar{q}\to{}W(Z)H$ (c), and associated
      production $gg\to{}t\bar{t}H$ (d).}
    \label{fig:higgs_production}}
\end{figure}

\begin{figure}[hbtp]
 \begin{center}
   \includegraphics[width=0.8\textwidth]{ch2_figs/higgs_prod_xsec.pdf}
   \caption[Higgs production cross section vs LHC collision energy]{Higgs production process cross section as a function of center-of-mass LHC collision energy~\cite{lhchxswg}.
     This dissertation analyzes events produced at 13 TeV.}
   \label{fig:higgs_prod_plot}
 \end{center}
\end{figure}


The decay modes of the Higgs are important in the context of an experimental collider search. The Higgs decay mode and to some extent, the production mode
determine which decay channels are most relevant for an experimental search. After being produced, the Higgs decays almost instantly\footnote{
The Higgs has lifetime of $10^{-22}s$~\cite{pdg}} to pairs of identical SM particles. 
The fraction of total Higgs decays that produce a given pair of SM particles, is referred to as the branching fraction. This branching fraction value is unique
to each set of final state particles the Higgs decays to.
The Higgs couples more strongly to massive particles and less strongly to lighter particles. This means a decay to heavy
particles is more likely than a decay to light particles. This is true with the caveat that this effect is balanced by the fact that the Higgs itself has a
mass of approximately 125 GeV, and decaying to a particle pair with mass greater than the Higgs mass is strongly suppressed. Decays to heavier
states, such as $H\rightarrow WW$ are allowed, but at least one W is produced off-shell, that is with a lighter mass. It is the decay to off-shell
particles that supresses the branching fraction. The greater the off-shell particle's mass differs from the pole mass, the larger the supression.
This explains some of the decay mode behavior as a function of Higgs mass in
Figure~\ref{fig:higgs_decay}.

\begin{figure}[hbtp]
 \begin{center}
   \includegraphics[width=0.8\textwidth]{ch2_figs/higgs_decay.pdf}
   \caption[Higgs branching fractions vs mass]{Higgs branching fractions as a function of Higgs mass. While the Higgs mass has now been measured
     to approximately 125 GeV, this illustrates how the branching fractions are affected by varying the Higgs mass.~\cite{lhchxswg}.}
   \label{fig:higgs_decay}
 \end{center}
\end{figure}

While the SM as described here paints a picture of a complete theory, many important questions remain, such as those posed in the introduction.
A thorough study and understanding of \tth processes can help address these questions.  

\section{ttH}
\label{sec:tth}
\subsection{Description}
In a broader theoretical context, \tth searches are in essence a probe of the SM, \emph{directly} testing the Yukawa coupling strength of the Higgs boson to top quarks. The fact that a
measurement of \tth is a direct probe is an important distinction between an indirect measurement. As mentioned previously, the Higgs was discovered and proven
consistent with the SM via the gluon fusion production mode in Figure~\ref{fig:higgs_production}. The gluon fusion diagram includes a fermion loop (triangle)
that produces the Higgs. Because the Higgs couples more strongly to massive particles, the top quark contribution dominates in this loop. And
because the Higgs was already confirmed and proven consistent with SM predictions being produced this way, the top-Higgs Yukawa coupling has already been
measured, albeit \emph{indirectly}. Beacuse we don't directly observe the fermions in this loop, it is possible that other particles beyond the SM contribute.
Enter \tth. While the primary goal of searching for \tth has been motivated,
other important questions about the SM can be addressed simultaneously. The top quark is unique with respect to all other quarks. It is the heaviest, but curiously
it is approximately 40x heavier than the next heaviest quark. The top quark mass
being so much greater than any other quark inspires questions about the pattern of the quark masses. Does the top quark's mass come only from the Higgs? Or could
it also come from something beyond the SM? Direct \tth searches will help answer these questions. 

%% \begin{figure}[hbtp]
%%  \begin{center}
%%    \includegraphics[width=0.4\textwidth]{ch2_figs/feynman_diagram_ttH_HWW_2lss.pdf}
%%    \caption{A \tth process at the LHC. The final state here is one of the targets of the analysis presented later in this dissertation.}
%%    \label{fig:tth_example_diagram}
%%  \end{center}
%% \end{figure}

In an experimental context, \tth is produced by two gluons, each connecting a top-antitop quark pair, where a top and an
antitop from each gluon connect, and the Higgs can be produced off of any (anti)top quark line, represented by the feynman diagram in Figure~\ref{fig:tth_example_diagram}.
The remaining top and antitop quark decay to a W boson and b quark with nearly 100$\%$ probability~\footnote{While the top can in principle decay
to quark flavors other than bottoms, these decays are so heavily CKM suppressed~\cite{pdg} that they are
neglected.}. The W boson produced in the top decays instantly to pairs of SM particles. These pairs include a charged lepton and a neutrino of the same
flavor approximately 1/3 of the time, while the rest of the time~\cite{pdg} the W decays to a quark-antiquark where the quark and antiquark have different flavors. 
Like the W, the Higgs is free to decay to numerous states according to Figure~\ref{fig:higgs_decay}. Because of the various decay modes of the Higgs and
the Ws there are many final states possible in \tth. These numerous possible final states dictate the experimental techniques employed in searches,
with completely separate analysis efforts dedicated to a single \tth higgs final state, or a closely related collection of Higgs final states.


\subsection{Multilepton Final States}
The specific signal targeted by this analysis is \tth decaying to final states with two or more charged leptons. Examples of this signal are in
Figure~\ref{fig:tth_example_diagram}. There are multiple Higgs decays included this definition, specificaly WW, ZZ and $\tau\tau$. The fractions of these Higgs decays
is in Figure~\ref{fig:pie}. 

\begin{figure}[htb]
\centering
\includegraphics[width=0.25\linewidth]{ch2_figs/gg-ttH-tt-2lss.pdf}
\hspace{0.5cm}
\includegraphics[width=0.25\linewidth]{ch2_figs/gg-ttH-ZZ-3l.pdf}
\hspace{0.5cm}
\includegraphics[width=0.25\linewidth]{ch2_figs/gg-ttH-WW-4l.pdf}
\caption[\tth feynman diagrams]{Examples of leading order Feynman diagrams for $t\bar{t}H$ production at pp colliders, with the Higgs boson decaying to
$\tau\tau$, $\mathrm{ZZ}^{*}$, and
$\mathrm{WW}^{*}$ (from left to right). The first, second, and third diagrams are examples of the two same-sign lepton signature,
the three lepton signature, and the four lepton signature, respectively.}
\label{fig:tth_example_diagram}
\end{figure}

\begin{figure}[hbtp]
 \begin{center}
   \includegraphics[width=0.6\textwidth]{ch2_figs/pie.pdf}
   \caption[Pie chart of Higgs branching fractions]{The Higgs branching fractions assuming a mass of 125 GeV. The slices removed are the decays targeted in this analysis.}
   \label{fig:pie}
 \end{center}
\end{figure}

% % uncomment the following lines,
% if using chapter-wise bibliography
%
% \bibliographystyle{ndnatbib}
% \bibliography{example}
