%
% Chapter 11
%

\chapter{SUMMARY}
\label{chap:summary}
%This is done by comparing the observed number of events in data to the expected number of events from background plus \tth signal with a Higgs boson of mass 125 GeV. 
This dissertation presents a complete measurement of the \tth signal strength, targeting the $WW^{*}$, $ZZ^{*}$, and $\tau\tau$ decays of the Higgs boson
in the two same-sign leptons channel at $\sqrt{s} = 13$ TeV with an integrated luminosity of 35.9 fb$^{-1}$.
The data analyzed corresponds to the full dataset collected by the CMS experiment in 2016.
This analysis represents the most precise measurement of \tth in the $2lss$ channel to-date.
The signal and background are estimated with MC, with the exception of the backgrounds due to fake leptons and charge mis-assignments,
which are estimated with data. A BDT discriminant is constructed using kinematic and reconstruction-based BDT scores as input variables.
This discriminant is binned and used to extract the signal via a maximum likelihood fit in each of the ten sub-categories of the signal region.
This analysis is the first of its kind to observe evidence for SM \tth above the 3$\sigma$ significance level and represents significant
improvements over previous iterations, thanks in part, to increased separation power provided by reconstruction BDTs that target the hadronic top
and the jets from the Higgs. This analysis directly probes the SM via the top-Higgs Yukawa coupling, and while a small excess is observed
in the $e\mu$ and $\mu\mu$ channels, the observation is consistent with the SM within 1$\sigma$. 

In addition to testing the SM, this analysis demonstrates the performance improvement offered from reconstruction-oriented BDTs. With the inclusion of the BDTs
which reconstruct the Higgs jets and the hadronic top portions of the \tth event as inputs to the final BDTs,
the discrimination power improves by nearly 10$\%$ from the ROC curves. These reconstruction-oriented techniques show promising results and could be useful
in other analyses with complicated final states.

Finally, the work presented here is included in a paper to be submitted to the journal Physical Review B. 
