%
% Chapter 11
%

\chapter{STATISTICAL METHODS}
\label{chap:stats}
The \tth signal process, as well as the background, is inherently subject to statistical fluctuations due to quantum mechanics.
Additional randomness is introduced in the measurement process. This inherent randomness means that simply counting the number of predicted
signal and background events, and counting the number of events observed from data and comparing the two numbers is not necessarily enough to infer the presence or
lack thereof of the \tth signal process. Instead, the degree to which the
number of events in the signal and background predictions agree, given their uncertainties, with the number of events observed in data, must be quantified.
Furthermore we must quantify the probability that the numbers we observe are representative of nature at large, and not a statistical fluctuation driven by randomness. 
We use the likelihood function as the basis for estimating the signal strength parameter via the maximum likelihood technique, and also for composing our test statistic when
calculating upper limits  on the signal strength parameter via the CLs method. 


\section{Maximum Likelihood Fit}
\label{sec:fit}
The likelihood function is defined as the probability density of the number of observed events, $N$, given the predicted number of signal-plus-background events, $\mus + b$,
where $\mu$ is the signal strength parameter we wish to estimate. $\mu$ is more generally referred to as the parameter of interest, since it is free to float to the value
which best fits the observation. The signal strength parameter is assumed to be a modifier on the cross section of the \tth signal process, and does not depend on repeated
observations or bins. The simplified likelihood, ignoring nuisance parameters for now, is written as a Poisson probability under the freqentist approach as:

\begin{equation}
\label{eqn:likelihood1}
\mathcal{L}(data|\mu) = P(N|\mu s+b) = \frac{(\mu s+b)^{N}e^{-(\mu s+b)}}{N!}
\end{equation}

\noindent The above expression holds for a single bin of events, however this analysis is performed in each subcategory of the signal
region for each bin of the final BDT discriminant. The likelihood for this analysis must be calcualted separately for each bin of each signal region category $i$, and
the corresponding number of observed events in data $n_{i}$, and predictions of signal and background, $s_{i}, b_{i}$ respectively:

\begin{equation}
\label{eqn:likelihood2}
\mathcal{L}(data|\mu) = \prod_{i=1} \frac{(\mu s_{i}+b_{i})^{n_{i}}e^{-(\mu s_{i}+b_{i})}}{n_{i}!}
\end{equation}

\noindent Now the uncertainties associated with the signal and background predictions must be accounted for in the form of nuisance parameters. The expected signal
and background yields are re-written $s \rightarrow s(\theta)$ and $b \rightarrow b(\theta)$ to depend on the set of nuisances $\theta$. The expected value of the nuisances
is $\tilde{\theta}$. The nuisances are parametrized as a PDF $P(\theta|\tilde{\theta})$. With Bayes' theorem, this PDF becomes:

\begin{equation}
\label{eqn:nuisance}
P(\theta|\tilde{\theta}) ~ P(\tilde{\theta}|\theta) \cdot \pi(\theta)
\end{equation}

In this context, $P(\theta|\tilde{\theta})$ is the posterior probability, or the degree to which we believe the true value of $\theta$ given our estimate $\tilde{\theta}$.
$P(\tilde{\theta}|\theta)$ is interpreted as the probability of obtaining the value $\tilde{\theta}$ when the true value is $\theta$. $\pi(\theta)$ is the prior.
In this analysis, the prior is a gaussian distribution for shape systematics, and a log-normal distribution for rate systematics. For gaussian and log-normal distributions,
the prior $\pi(\theta)$ does not vary significantly with $\theta$. We can solve Equation~\ref{eqn:nuisance} for $P(\tilde{\theta}|\theta)$, which constrains the nuisances
and the likelihood can be modified properly:

\begin{equation}
\label{eqn:likelihood3}
\mathcal{L}(data|\mu,\theta) = \prod_{i=1} \frac{(\mu s_{i}(\theta)+b_{i}(\theta))^{n_{i}}e^{-(\mu s_{i}(\theta)+b_{i}(\theta))}}{n_{i}!} \cdot P(\tilde{\theta}|\theta)
\end{equation}

\noindent The estimated value for $\mu$ is obtained from finding the values of $\mu$ and $\theta$ which maximize the likelihood, denoted as $\hat{\mu}$ and $\hat{\theta}$.
This technique is called the maximum likelihood estimation (MLE)~\cite{lista}.
In practice, we first take the negative log of the likelihood (NLL), and then find the minimum, since it simplifies
the procedure by turning the product in Equation~\ref{eqn:likelihood3} into a sum. $\hat{\mu}$ is referred to as the ``best-fit'' $\mu$, because it is the value which best fits the data.
This maximum likelihood procedure is also referred to generally as the ``fit''~\cite{CMS-AN-11-298}. 

\section{Upper Limits: CLs Method}
\label{sec:limit}
When an analysis lacks the sensitivity to detect a signal from background, setting uppper limits on the possible values of $\mu$ is useful. In setting upper
limits, $\mu$ is constrained as phase space is excluded. This strategy was used by the Higgs analyses at LEP and the Tevatron, and is the primary figure of merit
for an analysis with little sensitivity. The \tth analysis presented here does not fall into this category, although previous iterations lacked the sensitivity
to detect \tth in data. While this analysis does have sensitivity to \tth, upper limits are still set using the CLs method, also known as the
Modified Frequentist approach, which builds on the likelihood described in the previous section. 

Starting from the likelihood in Equation~\ref{eqn:likelihood3}, we construct a test statistic, $q_{\mu}$, based on the profile likelihood ratio\footnote{The ``profile'' in profile
likelihood ratio is due to the fact that the nusiance parameter values are constrained from a fit to data.}~\cite{AsymptoticLimits},
defined as:

\begin{equation}
\label{eqn:test_stat}
\tilde{q}_{\mu} = -ln \frac{\mathcal{L}(data|\mu,\hat{\theta_{\mu}})}{\mathcal{L}(data|\hat{\mu},\hat{\theta})},~~~~~0 \leq \hat{\mu} \leq{\mu}
\end{equation}

\noindent Where $\hat{\theta_{\mu}}$ is the value of the nusiances corresponding to $\mu$. $\mathcal{L}(data|\mu,\hat{\theta_{\mu}})$ is the maximized where $\mu$
is fixed to some value and $\hat{\theta_{\mu}}$ floats freely. The denominator, $\mathcal{L}(data|\hat{\mu},\hat{\theta})$, is the maximum likelihood obtained
previously where both parameters float. The lower bound $0 \leq \hat{\mu}$ ensures the signal is positive. The upper bound $\hat{\mu} \leq \mu$ imposes a one-sided
confidence interval, which practically means that upward fluctuations of data $\hat{\mu} > \mu$, where the hypothesis is $\mu$, cannot be considered as evidence
against the hypothesis.  

With the definition of the test statistic, we calculate $\tilde{q}_{\mu}^{obs}$, the observed test statistic value in data, by evaluating it for many different
values of $\mu$. Now the values for the nuisance parameters observed in data, $\hat{\theta_{0}^{obs}}$, $\hat{\theta_{\mu}^{obs}}$ are obtained from maximizing
the likelihoods under the background-only ($\mu=0$), and signal-plus-background hypotheses respectively. Next, two test statistic PDFs
$f(\tilde{q}_{\mu}|\mu,\hat{\theta_{\mu}^{obs}})$, and $f(\tilde{q}_{\mu}|0,\hat{\theta_{0}^{obs}})$ are constructed from psuedo-data
generated with MC toys, obtained from random sampling of nusiance parameter values from the fit to data at fixed $\mu$.  Next we calculate a p-value $p_{\mu}$, $p_{b}$
for the signal plug background and background-only hypotheses. The p-value $p_{\mu}$ represents the probability that the observed data is incompatible with the
signal-plus-background hypothesis, with signal strength $\mu$. As $p_{\mu}$ decreases, the more confident we are that the value of $\mu$ being evaluated is the
upper limit on the signal strength. The p-value $p_{b}$ is the probability for compatibility with the background-only hypothesis. As $p_{b}$ decreases, the probability
of a signal in the data increases. It is more convenient to work with $1-p_{b}$, since this value is the probability of incompatibility with the background-only hypothesis,
just as $p_{s}$ is a probability of incompatibility with the signal-plus-background hypothesis:

\begin{equation}
\label{eqn:pvalues1}
p_{\mu} = P(\tilde{q}_{\mu} \geq \tilde{q}_{\mu}^{obs}|signal+background) = \int_{\tilde{q}_{\mu}^{obs}}^{\infty} f(\tilde{q}_{\mu}|\mu,\hat{\theta_{\mu}^{obs}}) d\tilde{q}_{\mu}
\end{equation}

\begin{equation}
\label{eqn:pvalues2}
1- p_{b} = P(\tilde{q}_{\mu} \geq \tilde{q}_{\mu}^{obs}|background-only) = \int_{\tilde{q}_{0}^{obs}}^{\infty} f(\tilde{q}_{\mu}|0,\hat{\theta_{0}^{obs}}) d\tilde{q}_{\mu}
\end{equation}

\noindent We define confidence levels (CL) for each hypothesis, with $CL_{s+b} = p_{\mu}$, and $CL_{b} = 1-p_{b}$, with $CL_{s}$ being the ratio of the two:

\begin{equation}
\label{eqn:cls}
CL_{s}(\mu) = \frac{CL_{s+b}}{CL_{b}} = \frac{p_{\mu}}{1-p_{b}}
\end{equation}

\noindent Interpreting Equation~\ref{eqn:cls}, we can say that as the probability for incompatibility with the background-only hypothesis increases, and/or as the probability
for incompatibility with the signal-plus-background hypothesis decreases, $CL_{s}(\mu)$ will decrease, and we become more confident that the observed data is more consistent with
the signal-plus-background hypothesis than the background-only hypothesis. The observed 95$\%$ CL upper limit on $\mu$ is obtained by testing different values of decreasing $\mu$ and
calculating $CL_{s}(\mu)$, the upper limit is the value of $\mu$ for which $CL_{s}(\mu) = 0.05$. In general we say that for some $\mu$ corresponding to $CL_{s}(\mu) \leq \alpha$
the signal is excluded at the $1-\alpha$ CL. 

The expected upper limit is calculated by generating the background-only distribution from MC toys as described previously, for many psuedo experiments, calculating $CL_{s}$, $\mu^{95\% CL}$
for each distribution (psuedo experiment). Then a cumulative distribution of $\mu^{95\% CL}$ is constructed and the median expected value on the upper limit of $\mu$ is reported as the value
of $\mu^{95\% CL}$ for which the cross the 50$\%$ quantile\footnote{Technically, generating N toys for M psuedo experiments yields N$\cdot$M likelihood evaluations. Since the test-statistic
distributions for a given $\mu$ don't depend on the psuedo data, they are instead computed only once per $\mu$ value, and the total number of likelihood evaluations is proportional to N+M instead.}

Generating large statistics of toy MC to produce the test statistic distributions need for the method above is both time consuming and CPU intensive. This analysis uses an alternate method to
construct the test statistic distributions analytically, without the need for psuedo data, called the asymptotic approximation of the profile likelihood~\cite{AsymptoticLimits}.  
This procedure begins by removing the requirement that the signal be positive $\hat{\mu} > 0$ from the test statistic in Equation~\ref{eqn:test_stat}. From Wilks theorem
in the asymptotic regime, the test statistic will have half a $\chi^{2}$ distribution for one degree of freedom in the signal-plus-background hypothesis~\cite{wilks}. The value of $\mu$ for which
$\frac{1}{2}q_{\mu} = 1.92$ has the convenient property of corresponding to $CL_{s} = 0.05$. By imposing $\hat{\mu} > 0$, the asymptotic behavior of the test statistic PDF under the signal plus
background hypothesis does follow a well-defined distribution:

\begin{equation}
\label{eqn:asymptote}
f(\tilde{q}_{\mu}|\mu) = \frac{1}{2}\delta(\tilde{q}_{\mu}) + \begin{cases}
  \frac{e^{-\tilde{q}_{\mu}/2}}{\sqrt{8\pi~\tilde{q}_{\mu}}} & 0 < \tilde{q}_{\mu} \leq \mu^{2}/\sigma^{2} \\
  \frac{e^{\frac{\tilde{q}_{\mu}+\mu^{2}/\sigma^{2}}{8\mu^{2}/\sigma^{2}}}}{\sqrt{8\pi\mu^{2}/\sigma^{2}}} & \tilde{q}_{\mu} > \mu^{2}/\sigma^{2}
  \end{cases}
\end{equation}

\noindent where $\sigma^{2} = \mu^{2}/q_{\mu,A}$, and $q_{\mu,A}$ is known as the Asimov dataset\footnote{The Asimov dataset is named after the 1955 short story ``Franchise'' by Issac Asimov, where the 2008
U.S. election determined by a single vote of one person, who is said to represent the entire population.} with all nuisances set to their initial values. The test statistic distributions for
signal-plus-background and background-only hypotheses are constructed from Equation~\ref{eqn:asymptote} instead of from toy MC~\cite{CMS-AN-11-298}. 

\section{Significance}
To determine the significance of a result, we use the asymptotic approximation of the profile likelihood described above. We calculate the probability of an observation that is as or less compatible with the
data in the background-only hypothesis. This is the probability that the background randomly fluctuated to produce the observation in data. The lower this probability, the greater the significance.
This is expressed as a p-value, $p_{0}$:

\begin{equation}
\label{eqn:signif1}
p_{0} = P(q_{0} \geq q_{0}^{obs}) = \int_{q_{0}^{obs}}^{\infty} f(q_{0}|0,\hat{\theta_{0}^{obs}}) dq_{0}
\end{equation}

\noindent This p-value is converted into a significance, $Z$ (in units of standard deviations, $\sigma$) by integrating one side of the gaussian tail:

\begin{equation}
\label{eqn:signif2}
p_{0} = \int_{Z}^{\infty} \frac{1}{\sqrt{2\pi}}e^{-x^{2}/2} dx
\end{equation}

\noindent The value of $Z$ represents the number of standard deviations the background, assuming no signal, would have to fluctuate by to be consistent with the observation. 
The significance can also be approximated under the asymptotic profile likelihood with the test statistic defined in Eauation~\ref{eqn:test_stat} under the background-only hypothesis:

\begin{equation}
\label{eqn:signif3}
Z = \sqrt{q_{0}}
\end{equation}
